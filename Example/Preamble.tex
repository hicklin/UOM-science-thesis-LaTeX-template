%%%::::::::::::::::::::::::::::::scienceThesis V2::::::::::::::::::::::::::::::%%%
%%%:::::::::::::::::::::::::::::::::::::::::::::::::::::::::::::::::::::::::::::::
% Hello!!! Welcome to the \LaTeX{} template for your University of Malta, Faculty 
% of Science thesis. This template was created (hacked!) by 
% William Hicklin B.Sc (Hons) Chem & Phy and Eric Pace B.Sc (Hons) Phy & AI (2012)
%%%:::::::::::::::::::::::::::::::::::::::::::::::::::::::::::::::::::::::::::::::
% This is the preamble code required for part of the thesis type-setting. This
% code must be included before the \begin{document} in the scienceThesis.tex file.
% To do this use the command %%%::::::::::::::::::::::::::::::scienceThesis V2::::::::::::::::::::::::::::::%%%
%%%:::::::::::::::::::::::::::::::::::::::::::::::::::::::::::::::::::::::::::::::
% Hello!!! Welcome to the \LaTeX{} template for your University of Malta, Faculty 
% of Science thesis. This template was created (hacked!) by 
% William Hicklin B.Sc (Hons) Chem & Phy and Eric Pace B.Sc (Hons) Phy & AI (2012)
%%%:::::::::::::::::::::::::::::::::::::::::::::::::::::::::::::::::::::::::::::::
% This is the preamble code required for part of the thesis type-setting. This
% code must be included before the \begin{document} in the scienceThesis.tex file.
% To do this use the command %%%::::::::::::::::::::::::::::::scienceThesis V2::::::::::::::::::::::::::::::%%%
%%%:::::::::::::::::::::::::::::::::::::::::::::::::::::::::::::::::::::::::::::::
% Hello!!! Welcome to the \LaTeX{} template for your University of Malta, Faculty 
% of Science thesis. This template was created (hacked!) by 
% William Hicklin B.Sc (Hons) Chem & Phy and Eric Pace B.Sc (Hons) Phy & AI (2012)
%%%:::::::::::::::::::::::::::::::::::::::::::::::::::::::::::::::::::::::::::::::
% This is the preamble code required for part of the thesis type-setting. This
% code must be included before the \begin{document} in the scienceThesis.tex file.
% To do this use the command %%%::::::::::::::::::::::::::::::scienceThesis V2::::::::::::::::::::::::::::::%%%
%%%:::::::::::::::::::::::::::::::::::::::::::::::::::::::::::::::::::::::::::::::
% Hello!!! Welcome to the \LaTeX{} template for your University of Malta, Faculty 
% of Science thesis. This template was created (hacked!) by 
% William Hicklin B.Sc (Hons) Chem & Phy and Eric Pace B.Sc (Hons) Phy & AI (2012)
%%%:::::::::::::::::::::::::::::::::::::::::::::::::::::::::::::::::::::::::::::::
% This is the preamble code required for part of the thesis type-setting. This
% code must be included before the \begin{document} in the scienceThesis.tex file.
% To do this use the command \input{Preamble.tex}.
%%%:::::::::::::::::::::::::::::::::::::::::::::::::::::::::::::::::::::::::::::::
\documentclass[12pt,a4paper]{report}

\usepackage{packages/scienceThesis}% Needed for further type-settings and commands
\usepackage{verbatim}% needed for the ``code" and other environments
\usepackage{amsfonts}
\usepackage{amsmath}
\usepackage{amssymb}
\usepackage{amsthm}
\usepackage{bibunits}% needed for the bibliography section
\usepackage{lipsum}% needed for the generating text in the example
\usepackage[acronym,toc,nonumberlist]{glossaries}% This package is required for adding acronyms.


%%%:::::::::::::::::::::::::::::::::::::::::::::::::::::::::::::::::::::::::::::::
% The following environments are useful to present proofs in your thesis. These
% packages are not really necessary, if you don't need the code and proofs
% environments, so if you like you can delete from here till ``the next comment".
% Note that there are some examples below which obviously won't work once you
% remove this part
%%%:::::::::::::::::::::::::::::::::::::::::::::::::::::::::::::::::::::::::::::::
\theoremstyle{definition}
\newtheorem{definition}{Definition}[section]
\theoremstyle{definition}%plain}
\newtheorem{example}{Example}[section]
\theoremstyle{definition}%remark}
\newtheorem{proposition}{Proposition}[section]
\theoremstyle{definition}%remark}
\newtheorem{lemma}{Lemma}[section]
\theoremstyle{definition}%remark}
\newtheorem{corollary}{Corollary}[section]
\theoremstyle{definition}%remark}
\newtheorem{theorem}{Theorem}[section]
%%%:::::::::::::::::::::::::::::::::::::::::::::::::::::::::::::::::::::::::::::::
% The next comment!
%%%:::::::::::::::::::::::::::::::::::::::::::::::::::::::::::::::::::::::::::::::

%%%:::::::::::::::::::::::::::::::::::::::::::::::::::::::::::::::::::::::::::::::
% This section customises the headers of chapters, headers and footers.
%%%:::::::::::::::::::::::::::::::::::::::::::::::::::::::::::::::::::::::::::::::
% Customising headers and footers.
%:::::::::::::::::::::::::::::::::::::::::::::::::::::::::::::::::::::::::::::::::
\usepackage{fancyhdr}
\pagestyle{fancy}
\rhead{}
\lhead{\nouppercase{\textsc{\leftmark}}}
\renewcommand{\headrulewidth}{1pt}
\makeatletter
\renewcommand{\chaptermark}[1]{\markboth{\small\textsc{\@chapapp}\ \thechapter:\ \sc{#1}}{}}
\makeatother
%:::::::::::::::::::::::::::::::::::::::::::::::::::::::::::::::::::::::::::::::::
% Customising chapter headings.
%:::::::::::::::::::::::::::::::::::::::::::::::::::::::::::::::::::::::::::::::::
\usepackage{sectsty}
\chapterfont{\large\sc\centering}
\chaptertitlefont{\sc\centering}
\subsubsectionfont{\centering}
%%%:::::::::::::::::::::::::::::::::::::::::::::::::::::::::::::::::::::::::::::::


%%%:::::::::::::::::::::::::::::::::::::::::::::::::::::::::::::::::::::::::::::::
% PDF hyper-linking (set colours to black for printing)
%%%:::::::::::::::::::::::::::::::::::::::::::::::::::::::::::::::::::::::::::::::
\usepackage[colorlinks]{hyperref}
%\usepackage[figure,table]{hypcap}
\hypersetup{
	bookmarksnumbered,
	pdfstartview={FitH},
	citecolor={black},
	linkcolor={black},
	urlcolor={black},
	pdfpagemode={UseOutlines}
}
%%%:::::::::::::::::::::::::::::::::::::::::::::::::::::::::::::::::::::::::::::::
\setlength{\headheight}{15pt}


%%%:::::::::::::::::::::::::::::::::::::::::::::::::::::::::::::::::::::::::::::::
% Sets the document to be line separated paragraphs not indentation separated
% paragraphs. To change to an indentation with no line skip simply change
% \parindent 0cm --> \parindent 1.3cm & \parskip 2ex --> \parskip 0ex
%%%:::::::::::::::::::::::::::::::::::::::::::::::::::::::::::::::::::::::::::::::
\parindent 0cm
\parskip 2ex

\makeglossaries

%%%:::::::::::::::::::::::::::::::::::::::::::::::::::::::::::::::::::::::::::::::
% End of preamble
%%%:::::::::::::::::::::::::::::::::::::::::::::::::::::::::::::::::::::::::::::::.
%%%:::::::::::::::::::::::::::::::::::::::::::::::::::::::::::::::::::::::::::::::
\documentclass[12pt,a4paper]{report}

\usepackage{packages/scienceThesis}% Needed for further type-settings and commands
\usepackage{verbatim}% needed for the ``code" and other environments
\usepackage{amsfonts}
\usepackage{amsmath}
\usepackage{amssymb}
\usepackage{amsthm}
\usepackage{bibunits}% needed for the bibliography section
\usepackage{lipsum}% needed for the generating text in the example
\usepackage[acronym,toc,nonumberlist]{glossaries}% This package is required for adding acronyms.


%%%:::::::::::::::::::::::::::::::::::::::::::::::::::::::::::::::::::::::::::::::
% The following environments are useful to present proofs in your thesis. These
% packages are not really necessary, if you don't need the code and proofs
% environments, so if you like you can delete from here till ``the next comment".
% Note that there are some examples below which obviously won't work once you
% remove this part
%%%:::::::::::::::::::::::::::::::::::::::::::::::::::::::::::::::::::::::::::::::
\theoremstyle{definition}
\newtheorem{definition}{Definition}[section]
\theoremstyle{definition}%plain}
\newtheorem{example}{Example}[section]
\theoremstyle{definition}%remark}
\newtheorem{proposition}{Proposition}[section]
\theoremstyle{definition}%remark}
\newtheorem{lemma}{Lemma}[section]
\theoremstyle{definition}%remark}
\newtheorem{corollary}{Corollary}[section]
\theoremstyle{definition}%remark}
\newtheorem{theorem}{Theorem}[section]
%%%:::::::::::::::::::::::::::::::::::::::::::::::::::::::::::::::::::::::::::::::
% The next comment!
%%%:::::::::::::::::::::::::::::::::::::::::::::::::::::::::::::::::::::::::::::::

%%%:::::::::::::::::::::::::::::::::::::::::::::::::::::::::::::::::::::::::::::::
% This section customises the headers of chapters, headers and footers.
%%%:::::::::::::::::::::::::::::::::::::::::::::::::::::::::::::::::::::::::::::::
% Customising headers and footers.
%:::::::::::::::::::::::::::::::::::::::::::::::::::::::::::::::::::::::::::::::::
\usepackage{fancyhdr}
\pagestyle{fancy}
\rhead{}
\lhead{\nouppercase{\textsc{\leftmark}}}
\renewcommand{\headrulewidth}{1pt}
\makeatletter
\renewcommand{\chaptermark}[1]{\markboth{\small\textsc{\@chapapp}\ \thechapter:\ \sc{#1}}{}}
\makeatother
%:::::::::::::::::::::::::::::::::::::::::::::::::::::::::::::::::::::::::::::::::
% Customising chapter headings.
%:::::::::::::::::::::::::::::::::::::::::::::::::::::::::::::::::::::::::::::::::
\usepackage{sectsty}
\chapterfont{\large\sc\centering}
\chaptertitlefont{\sc\centering}
\subsubsectionfont{\centering}
%%%:::::::::::::::::::::::::::::::::::::::::::::::::::::::::::::::::::::::::::::::


%%%:::::::::::::::::::::::::::::::::::::::::::::::::::::::::::::::::::::::::::::::
% PDF hyper-linking (set colours to black for printing)
%%%:::::::::::::::::::::::::::::::::::::::::::::::::::::::::::::::::::::::::::::::
\usepackage[colorlinks]{hyperref}
%\usepackage[figure,table]{hypcap}
\hypersetup{
	bookmarksnumbered,
	pdfstartview={FitH},
	citecolor={black},
	linkcolor={black},
	urlcolor={black},
	pdfpagemode={UseOutlines}
}
%%%:::::::::::::::::::::::::::::::::::::::::::::::::::::::::::::::::::::::::::::::
\setlength{\headheight}{15pt}


%%%:::::::::::::::::::::::::::::::::::::::::::::::::::::::::::::::::::::::::::::::
% Sets the document to be line separated paragraphs not indentation separated
% paragraphs. To change to an indentation with no line skip simply change
% \parindent 0cm --> \parindent 1.3cm & \parskip 2ex --> \parskip 0ex
%%%:::::::::::::::::::::::::::::::::::::::::::::::::::::::::::::::::::::::::::::::
\parindent 0cm
\parskip 2ex

\makeglossaries

%%%:::::::::::::::::::::::::::::::::::::::::::::::::::::::::::::::::::::::::::::::
% End of preamble
%%%:::::::::::::::::::::::::::::::::::::::::::::::::::::::::::::::::::::::::::::::.
%%%:::::::::::::::::::::::::::::::::::::::::::::::::::::::::::::::::::::::::::::::
\documentclass[12pt,a4paper]{report}

\usepackage{packages/scienceThesis}% Needed for further type-settings and commands
\usepackage{verbatim}% needed for the ``code" and other environments
\usepackage{amsfonts}
\usepackage{amsmath}
\usepackage{amssymb}
\usepackage{amsthm}
\usepackage{bibunits}% needed for the bibliography section
\usepackage{lipsum}% needed for the generating text in the example
\usepackage[acronym,toc,nonumberlist]{glossaries}% This package is required for adding acronyms.


%%%:::::::::::::::::::::::::::::::::::::::::::::::::::::::::::::::::::::::::::::::
% The following environments are useful to present proofs in your thesis. These
% packages are not really necessary, if you don't need the code and proofs
% environments, so if you like you can delete from here till ``the next comment".
% Note that there are some examples below which obviously won't work once you
% remove this part
%%%:::::::::::::::::::::::::::::::::::::::::::::::::::::::::::::::::::::::::::::::
\theoremstyle{definition}
\newtheorem{definition}{Definition}[section]
\theoremstyle{definition}%plain}
\newtheorem{example}{Example}[section]
\theoremstyle{definition}%remark}
\newtheorem{proposition}{Proposition}[section]
\theoremstyle{definition}%remark}
\newtheorem{lemma}{Lemma}[section]
\theoremstyle{definition}%remark}
\newtheorem{corollary}{Corollary}[section]
\theoremstyle{definition}%remark}
\newtheorem{theorem}{Theorem}[section]
%%%:::::::::::::::::::::::::::::::::::::::::::::::::::::::::::::::::::::::::::::::
% The next comment!
%%%:::::::::::::::::::::::::::::::::::::::::::::::::::::::::::::::::::::::::::::::

%%%:::::::::::::::::::::::::::::::::::::::::::::::::::::::::::::::::::::::::::::::
% This section customises the headers of chapters, headers and footers.
%%%:::::::::::::::::::::::::::::::::::::::::::::::::::::::::::::::::::::::::::::::
% Customising headers and footers.
%:::::::::::::::::::::::::::::::::::::::::::::::::::::::::::::::::::::::::::::::::
\usepackage{fancyhdr}
\pagestyle{fancy}
\rhead{}
\lhead{\nouppercase{\textsc{\leftmark}}}
\renewcommand{\headrulewidth}{1pt}
\makeatletter
\renewcommand{\chaptermark}[1]{\markboth{\small\textsc{\@chapapp}\ \thechapter:\ \sc{#1}}{}}
\makeatother
%:::::::::::::::::::::::::::::::::::::::::::::::::::::::::::::::::::::::::::::::::
% Customising chapter headings.
%:::::::::::::::::::::::::::::::::::::::::::::::::::::::::::::::::::::::::::::::::
\usepackage{sectsty}
\chapterfont{\large\sc\centering}
\chaptertitlefont{\sc\centering}
\subsubsectionfont{\centering}
%%%:::::::::::::::::::::::::::::::::::::::::::::::::::::::::::::::::::::::::::::::


%%%:::::::::::::::::::::::::::::::::::::::::::::::::::::::::::::::::::::::::::::::
% PDF hyper-linking (set colours to black for printing)
%%%:::::::::::::::::::::::::::::::::::::::::::::::::::::::::::::::::::::::::::::::
\usepackage[colorlinks]{hyperref}
%\usepackage[figure,table]{hypcap}
\hypersetup{
	bookmarksnumbered,
	pdfstartview={FitH},
	citecolor={black},
	linkcolor={black},
	urlcolor={black},
	pdfpagemode={UseOutlines}
}
%%%:::::::::::::::::::::::::::::::::::::::::::::::::::::::::::::::::::::::::::::::
\setlength{\headheight}{15pt}


%%%:::::::::::::::::::::::::::::::::::::::::::::::::::::::::::::::::::::::::::::::
% Sets the document to be line separated paragraphs not indentation separated
% paragraphs. To change to an indentation with no line skip simply change
% \parindent 0cm --> \parindent 1.3cm & \parskip 2ex --> \parskip 0ex
%%%:::::::::::::::::::::::::::::::::::::::::::::::::::::::::::::::::::::::::::::::
\parindent 0cm
\parskip 2ex

\makeglossaries

%%%:::::::::::::::::::::::::::::::::::::::::::::::::::::::::::::::::::::::::::::::
% End of preamble
%%%:::::::::::::::::::::::::::::::::::::::::::::::::::::::::::::::::::::::::::::::.
%%%:::::::::::::::::::::::::::::::::::::::::::::::::::::::::::::::::::::::::::::::
\documentclass[12pt,a4paper]{report}

\usepackage{packages/scienceThesis}% Needed for further type-settings and commands
\usepackage{verbatim}% needed for the ``code" and other environments
\usepackage{amsfonts}
\usepackage{amsmath}
\usepackage{amssymb}
\usepackage{amsthm}
\usepackage{bibunits}% needed for the bibliography section
\usepackage{lipsum}% needed for the generating text in the example
\usepackage[acronym,toc,nonumberlist]{glossaries}% This package is required for adding acronyms.


%%%:::::::::::::::::::::::::::::::::::::::::::::::::::::::::::::::::::::::::::::::
% The following environments are useful to present proofs in your thesis. These
% packages are not really necessary, if you don't need the code and proofs
% environments, so if you like you can delete from here till ``the next comment".
% Note that there are some examples below which obviously won't work once you
% remove this part
%%%:::::::::::::::::::::::::::::::::::::::::::::::::::::::::::::::::::::::::::::::
\theoremstyle{definition}
\newtheorem{definition}{Definition}[section]
\theoremstyle{definition}%plain}
\newtheorem{example}{Example}[section]
\theoremstyle{definition}%remark}
\newtheorem{proposition}{Proposition}[section]
\theoremstyle{definition}%remark}
\newtheorem{lemma}{Lemma}[section]
\theoremstyle{definition}%remark}
\newtheorem{corollary}{Corollary}[section]
\theoremstyle{definition}%remark}
\newtheorem{theorem}{Theorem}[section]
%%%:::::::::::::::::::::::::::::::::::::::::::::::::::::::::::::::::::::::::::::::
% The next comment!
%%%:::::::::::::::::::::::::::::::::::::::::::::::::::::::::::::::::::::::::::::::

%%%:::::::::::::::::::::::::::::::::::::::::::::::::::::::::::::::::::::::::::::::
% This section customises the headers of chapters, headers and footers.
%%%:::::::::::::::::::::::::::::::::::::::::::::::::::::::::::::::::::::::::::::::
% Customising headers and footers.
%:::::::::::::::::::::::::::::::::::::::::::::::::::::::::::::::::::::::::::::::::
\usepackage{fancyhdr}
\pagestyle{fancy}
\rhead{}
\lhead{\nouppercase{\textsc{\leftmark}}}
\renewcommand{\headrulewidth}{1pt}
\makeatletter
\renewcommand{\chaptermark}[1]{\markboth{\small\textsc{\@chapapp}\ \thechapter:\ \sc{#1}}{}}
\makeatother
%:::::::::::::::::::::::::::::::::::::::::::::::::::::::::::::::::::::::::::::::::
% Customising chapter headings.
%:::::::::::::::::::::::::::::::::::::::::::::::::::::::::::::::::::::::::::::::::
\usepackage{sectsty}
\chapterfont{\large\sc\centering}
\chaptertitlefont{\sc\centering}
\subsubsectionfont{\centering}
%%%:::::::::::::::::::::::::::::::::::::::::::::::::::::::::::::::::::::::::::::::


%%%:::::::::::::::::::::::::::::::::::::::::::::::::::::::::::::::::::::::::::::::
% PDF hyper-linking (set colours to black for printing)
%%%:::::::::::::::::::::::::::::::::::::::::::::::::::::::::::::::::::::::::::::::
\usepackage[colorlinks]{hyperref}
%\usepackage[figure,table]{hypcap}
\hypersetup{
	bookmarksnumbered,
	pdfstartview={FitH},
	citecolor={black},
	linkcolor={black},
	urlcolor={black},
	pdfpagemode={UseOutlines}
}
%%%:::::::::::::::::::::::::::::::::::::::::::::::::::::::::::::::::::::::::::::::
\setlength{\headheight}{15pt}


%%%:::::::::::::::::::::::::::::::::::::::::::::::::::::::::::::::::::::::::::::::
% Sets the document to be line separated paragraphs not indentation separated
% paragraphs. To change to an indentation with no line skip simply change
% \parindent 0cm --> \parindent 1.3cm & \parskip 2ex --> \parskip 0ex
%%%:::::::::::::::::::::::::::::::::::::::::::::::::::::::::::::::::::::::::::::::
\parindent 0cm
\parskip 2ex

\makeglossaries

%%%:::::::::::::::::::::::::::::::::::::::::::::::::::::::::::::::::::::::::::::::
% End of preamble
%%%:::::::::::::::::::::::::::::::::::::::::::::::::::::::::::::::::::::::::::::::